\usepackage[utf8]{inputenc}
\usepackage[T1]{fontenc}
% \usepackage[ruled, linesnumbered]{algorithm2e} % ,oldcommands]{algorithm2e} % 2013-05-15 Michael: For algorithms. 2013-05-24 Michael: I can't compile with options oldcommand.
\usepackage[ruled, linesnumbered, oldcommands]{algorithm2e}
\usepackage{graphicx}
\usepackage{subfigure}
\usepackage{amsmath}
% \usepackage{cleveref}
\usepackage{amsfonts}
\usepackage{color}
% \usepackage{algorithmic}
\usepackage[numbered,framed]{mcode}
\usepackage{multirow}
\usepackage{charter}
\usepackage{hyperref}
\usepackage{tikz}
\usetikzlibrary{shapes,arrows}
\usepackage{todonotes}
\usepackage{morefloats}  % Get rid of LaTex too many unprocessed floats problem

\def\registered{\textsuperscript{\textregistered}}
\def\copyright{\textsuperscript{\textcopyright}}
\def\trademark{\textsuperscript{\texttrademark}}

\definecolor{kennycolor}{rgb}{0.7,0.1,0.1}
\newcommand{\kenny}[1]{  { \color{kennycolor} Kenny says: #1}    }
\definecolor{michaelcolor}{rgb}{0.1,0.1,0.7}
\newcommand{\michael}[1]{  { \color{michaelcolor} Michael says: #1}    }

\newtheorem{theorem}{Theorem}[section]
\newtheorem{lemma}[theorem]{Lemma}
\newtheorem{proposition}[theorem]{Proposition}
\newtheorem{corollary}[theorem]{Corollary}
\newtheorem{definition}{Definition}[section]
% \newtheorem{algorithm}{Algorithm}[section]

\newenvironment{proof}[1][Proof]{\begin{trivlist}
\item[\hskip \labelsep {\bfseries #1}]}{\end{trivlist}}
\newenvironment{example}[1][Example]{\begin{trivlist}
\item[\hskip \labelsep {\bfseries #1}]}{\end{trivlist}}
\newenvironment{remark}[1][Remark]{\begin{trivlist}
\item[\hskip \labelsep {\bfseries #1}]}{\end{trivlist}}
\newenvironment{exercise}[1][Exercise]{\begin{trivlist}
\item[\hskip \labelsep {\bfseries #1}]}{\end{trivlist}}
\newenvironment{answer}[1][Answer]{\begin{trivlist}
\item[\hskip \labelsep {\bfseries #1}]}{\end{trivlist}}

\newcommand{\qed}{\nobreak \ifvmode \relax \else
      \ifdim\lastskip<1.5em \hskip-\lastskip
      \hskip1.5em plus0em minus0.5em \fi \nobreak
      \vrule height0.75em width0.5em depth0.25em\fi}

\renewcommand{\vec}[1]{ \ensuremath{\mathbf{#1} } }
\newcommand{\abs}[1]{\left| {#1} \right|}
\newcommand{\quat}[1]{ #1 }
\newcommand{\mat}[1]{\ensuremath{\mathbf{#1} }}
\newcommand{\norm}[1]{\parallel {#1} \parallel}
\newcommand{\N}{ \mathbb{N} }
\renewcommand{\Re}{ \mathbb{R} }
\newcommand{\set}[1]{\mathcal{#1}}
\newcommand{\identity}{ \ensuremath{ \mat I }}
\newcommand{\bigO}{ \ensuremath{ \mathcal{O} }}
\newcommand{\atan }{ \ensuremath{ \phantom{\cdot}\text{atan}_2 } }
\newcommand{\acos }{ \ensuremath{ \cos^{-1} } }
\newcommand{\asin }{ \ensuremath{ \sin^{-1} } }
\newcommand{\logand }{ \ensuremath{ \wedge } }
\newcommand{\logor }{ \ensuremath{ \vee } }
\newcommand{\degree}{\,^{\circ}}
\renewcommand{\th}{ \ensuremath{ ^{\text{th}}} }
\newcommand{\sgn }[1]{ \ensuremath{ \text{sgn}\left( #1 \right) } }
\newcommand{\union}{\ensuremath{ \cup }}
\newcommand{\intersection}{\ensuremath{ \cap }}
\newcommand{\hull }[1]{ \ensuremath{ \text{convex hull}\left( #1 \right) } }
\newcommand{\proj }[1]{ \ensuremath{ \text{proj}\left( #1 \right) } }
\newcommand{\trace }[1]{ \ensuremath{ \text{tr}\left( #1 \right) } }
\renewcommand{\det}[1]{ \ensuremath{ \text{det}\left( #1 \right) } }
\newcommand{\idx }[1]{ \ensuremath{ \mathbb{#1} } }
\newcommand{\prox }[2]{ \ensuremath{ \text{prox}_{#1}\left( #2 \right) } }
